\documentclass[aspectratio=169]{beamer}
\usetheme[]{metropolis}           % Use metropolis theme;
%\usepackage{FiraSans}

\usepackage{appendixnumberbeamer}
\usepackage[absolute,overlay]{textpos}
\usepackage[T1]{fontenc}
\usepackage{setspace}
\usepackage{xcolor}
\usepackage{caption}

%%%%%%%%%%%%%%%%%%%%%%%
% Background setup
\pgfdeclareimage[width=\paperwidth,height=\paperheight]{bg2}{figs/bg2} % Declare background image

\defbeamertemplate*{background canvas}{mydefault}
  {%
    \ifbeamercolorempty[bg]{background canvas}{}{\color{bg}\vrule width\paperwidth height\paperheight}% copied beamer default here
  }
  \defbeamertemplate*{background canvas}{bg}
 {%
   \color{mDarkTeal}\vrule width\paperwidth height\paperheight% added bg color
 }

 \defbeamertemplate*{background canvas}{image}
{%
    \begin{tikzpicture}
        \useasboundingbox (0,0) rectangle (\the\paperwidth, \the\paperheight);
        \pgftext[at=\pgfpoint{0cm}{0cm}, left, base]{\pgfuseimage{bg2}};
    \end{tikzpicture}
}

 \BeforeBeginEnvironment{frame}{%
   \setbeamertemplate{background canvas}[mydefault]%
 }

 \makeatletter
 \define@key{beamerframe}{bg}[true]{%
   \setbeamertemplate{background canvas}[bg]%
 }

\makeatletter
\define@key{beamerframe}{image}[true]{%
  \setbeamercovered{invisible}%
  \setbeamertemplate{background canvas}[image]%
}

%%%%%%%%%%%%%%%%%%%%%%
%% General setup
%% Fonts and typography
\setbeamerfont{title}{size=\huge,%
                      series=\bfseries}
\setbeamerfont*{subtitle}{size=\Large,%
                          series=\bfseries}
\setbeamerfont{author}{size=\large}
\setbeamerfont{date}{size=\large}
\setbeamerfont{frametitle}{size=\Large,%
                           series=\bfseries}

% color
\setbeamercolor{titlelike}{use=palette primary, parent=palette primary}
\setbeamercolor{author}{use=palette primary, parent=palette primary}
\setbeamercolor{date}{use=palette primary, parent=palette primary}
\setbeamercolor{institute}{use=palette primary, parent=palette primary}
\setbeamercolor{structure}{use=palette primary, parent=palette primary}

\setbeamertemplate{title}{
  \flushleft%
  \setstretch{.85}%
  \inserttitle%
  \par%
  \vspace*{0.1em}
 }
 \setbeamertemplate{subtitle}{
  \raggedright%
  \setstretch{.85}%
  \insertsubtitle%
  \par%
  \vspace*{0.5em}
}

% Title page frame
\setbeamertemplate{title page}{
%\pgfdeclareimage[height=0.6cm]{logo}{UdeS_blanc} % Declare first Logo
%\pgfdeclareimage[height=0.45cm]{logo2}{IELab} % Declare second Logo
  \begin{minipage}[b][\paperheight]{10cm}
    \ifx\inserttitlegraphic\@empty\else\usebeamertemplate*{title graphic}\fi
    \vfill%
    \ifx\inserttitle\@empty\else\usebeamertemplate*{title}\fi
    \ifx\insertsubtitle\@empty\else\usebeamertemplate*{subtitle}\fi
    %\usebeamertemplate*{title separator} % puts a line under the title
    \ifx\beamer@shortauthor\@empty\else\usebeamertemplate*{author}\fi
    \ifx\insertdate\@empty\else\usebeamertemplate*{date}\fi
    \ifx\insertinstitute\@empty\else\usebeamertemplate*{institute}\fi
    %\vfill%
    \vspace*{8mm}
    %\pgfuseimage{logo} % Use logo
    %\newcommand\mybar{\kern1pt\rule[-\dp\strutbox]{.8pt}{.7cm}{\hspace{2mm}}\kern1pt} % White bar
    %\textcolor{white}{\mybar{\pgfuseimage{logo2}}} % Second logo
    %\vfill
    \vspace*{15mm}
  \end{minipage}
}

%%%%%%%%%%%%%%%%%%%%%%%%%%%
% Examples

%% Makes title slide with only background defined color
  %\begin{frame}[bg]
    %\titlepage
  %\end{frame}
%% Makes a title slide with background + image
  %\begin{frame}[image]
    %\titlepage
  %\end{frame}
%% Makes also a title slide with background + image
  %\maketitle

%%%%%%%%%%%%%%%%%%%%%%%%%%%

\title{Aire de distribution et changements climatiques:}
\subtitle{Comment les interactions biotiques moduleront-elles la réponse?}
%\date{\today}
\author{Victor Cameron}




\begin{document}

%% Effect of climate change on the co-distribution of forest trees and passerine birds
\maketitle % make title produces formated title slide

%%% keys
%1. Le titre
%2. En écologie, spécialement avec les c.c. nous sommes intéressés à comprendre ce qui détermine la distribution des espèces
%3. L'aire de distribution d'une espèce est limitée
%4. Une connaissance des processus limitant l'aire de distribution est fondamentale pour prévoir les distributions dans le futur

%%%%%%%%%%%%%%%%%%%%%%%%%%%%%%%%%%%%%%%%%%%%%%%%%%%%%%%%%%%%%%%%%%%%%%%%%%%%%%%%%
%%% Contexte

%{%
%\setbeamertemplate{frame footer}{Ferrarini \textit{et al}. 2017}
%\begin{frame}{Contexte\hskip 1em \mdseries{\textcolor{lightgray}{Les limites de distribution}}}

%  Un \textbf{déplacement des aires de distribution} est attendu dans les 100 prochaines années

%  \begin{figure}
%  	\includegraphics[height=.60\paperheight]{figs/Ferrarini.png}
%    %\caption*{ \textbf{Eveloppe climatique de l'érable à sucre présente et projetée (2071-2100)}}
%  \end{figure}

%\end{frame}
%}

%%% L'aire de distribution d'une espèce est limitée
%%% Une connaissance des processus limitant l'aire de distribution est fondamentale pour prévoir les distributions dans le futur

%%%%%%%%%%%%%%%%%%%%%%%%%%%%%%%%%%%%%%%%%%%%%%%%%%%%%%%%%%%%%%%%%%%%%%%%%%%%%%%%%
%%% Contexte

{%
\setbeamertemplate{frame footer}{}
\begin{frame}{Contexte\hskip 1em \mdseries{\textcolor{lightgray}{Les limites de distribution}}}

  L'emplacement des limites de distribution est \textbf{sensible au climat}

  \begin{figure}
  	\includegraphics[height=.65\paperheight]{figs/corridor.jpg}
    %\caption*{ \textbf{Distribution présente et projetée}}
  \end{figure}

\end{frame}
}

%%% On s'attend à trouver les espèces là où le climat est favorable et absentes là où le climat est défavorable
%%%

%%%%%%%%%%%%%%%%%%%%%%%%%%%%%%%%%%%%%%%%%%%%%%%%%%%%%%%%%%%%%%%%%%%%%%%%%%%%%%%%%
%%% Contexte

{%
\setbeamertemplate{frame footer}{Ferrarini \textit{et al}. 2017}
\begin{frame}{Contexte\hskip 1em \mdseries{\textcolor{lightgray}{Projection des futures distributions}}}

  On s'attend à ce que les enveloppes climatiques se \textbf{déplacent vers le nord ou vers des altitudes plus élevées}

  \begin{figure} % Photos de passeraux associés à divers types de forêts
    % Paruline azurée (ou couronnée?) est associée aux érablières
    \includegraphics[width=0.5\textwidth]{figs/Ferrarini.png}
  \end{figure}

\end{frame}
}

%%% Les espèces réactives pourront suivre le déplacement de leur enveloppe climatique
%%% Les espèces moins réactives accuseront un retard
%%% Il a été observé que la végétation accuse un retard sur le déplacement son enveloppe climatique
%%% Différents temps de réactions peuvent faire que des espèces qui co-occurent présentement ne co-occurent plus dans le futur

%%%%%%%%%%%%%%%%%%%%%%%%%%%%%%%%%%%%%%%%%%%%%%%%%%%%%%%%%%%%%%%%%%%%%%%%%%%%%%%%%
%%% Contexte

{%
\setbeamertemplate{frame footer}{McKenney \textit{et al}. 2007, Talluto \textit{et al}. 2017}
\begin{frame}[t]{Contexte\hskip 1em \mdseries{\textcolor{lightgray}{Difficultés reliées au contexte}}}

  Les \textbf{modèles de distribution d'espèces} (SDMs) sont des modèles mathématiques qui corrèlent la \emph{distribution} d'une espèce avec des \emph{données climatiques}

  \begin{columns}[]
    \begin{column}{0.5\textwidth}
      \begin{figure}
        \includegraphics[height=.65\textwidth]{figs/Talluto2.png}
      \end{figure}
    \end{column}

    \begin{column}{0.5\textwidth}
      \includegraphics[height=.65\textwidth]{figs/mckenney.png}
    \end{column}
  \end{columns}

\end{frame}
}

%% Les SDMs sontdes modèles mathématiques

%%%%%%%%%%%%%%%%%%%%%%%%%%%%%%%%%%%%%%%%%%%%%%%%%%%%%%%%%%%%%%%%%%%%%%%%%%%%%%%%%
%%% Contexte

\begin{frame}{Contexte\hskip 1em \mdseries{\textcolor{lightgray}{Difficultés reliées au contexte}}}

  Les \textbf{modèles de distribution d'espèces} (SDMs) font de nombreuses suppositions:
    \begin{itemize}
      \item Distribution à l'équilibre avec l'enveloppe climatique;
      \item Absence de démographie;
      \item Absence de limite de dispersion;
      \item Absence d'interaction entre espèces;
      \item Réponse linéaire et instantannée aux changements climatiques.
    \end{itemize}

\end{frame}

%% Les outils actuels pour prédire l'impact des changements climatiques sur la distribution des espèces sont limités par l'absence de processus écologiques fondamentaux.
%% Def interaction: l'effet qu'une espèce  a sur le fitness d'une autre espèce (taux de croissance de population): pred-proie, habitat-utilisateur

%%%%%%%%%%%%%%%%%%%%%%%%%%%%%%%%%%%%%%%%%%%%%%%%%%%%%%%%%%%%%%%%%%%%%%%%%%%%%%%%%
% % % Contexte

\begin{frame}{Contexte\hskip 1em \mdseries{\textcolor{lightgray}{Difficultés reliées au contexte}}}

  %% Quand on s'intéresse au changements d'aire de répartition d'une communauté, certains processus sont à prendre en considérations:
  Les espèces qui \textbf{co-occurent}:
  \begin{itemize}
    \item Ont différents temps de réponse aux changements climatiques;
    \item Ne se reproduisent pas au même rythme;
    \item N'ont pas toutes la même capacité de dispersion;
    \item Interagissent.
  \end{itemize}

  \alert{Ces processus peuvent modifier la relation entre le climat et la distribution des espèces}

\end{frame}

%% Les processus peuvent avoir des conséquences inatendues sur les futures distributions

%%%%%%%%%%%%%%%%%%%%%%%%%%%%%%%%%%%%%%%%%%%%%%%%%%%%%%%%%%%%%%%%%%%%%%%%%%%%%%%%%
% % %

\begin{frame}{Objectifs\hskip 1em \mdseries{\textcolor{lightgray}{Subtitle}}}

  \textbf{Objectif général:} Évaluer les impacts des changements climatiques sur la distribution régionale d'une espèce en interaction avec son habitat.

  \textbf{\\Objectifs secondaires:}
  \begin{enumerate}
    \item Développer un outil théorique pour améliorer la prédiction des impacts des changements climatiques sur la distribution des espèces;
    \item Évaluer l'impact des interactions biotiques sur la réaction des aires de distribution aux changements climatiques.
  \end{enumerate}

\end{frame}

%% Structure spéciale: Étude théorique = exploration
%% Pas d'hypothèse

%%%%%%%%%%%%%%%%%%%%%%%%%%%%%%%%%%%%%%%%%%%%%%%%%%%%%%%%%%%%%%%%%%%%%%%%%%%%%%%%%
% % %

{%
\setbeamertemplate{frame footer}{Pulliam 2000, Sandoro 2008}
\begin{frame}{Théorie \hskip 1em \mdseries{\textcolor{lightgray}{Interactions biotiques}}}

Les \textbf{interactions biotiques} sont d'importantes forces modulaires des limites de distribution à \emph{petites et à grandes échelles spatiales}

\begin{figure}
  \includegraphics[height=.50\paperheight]{figs/Sandoro.png}
  \caption*{Distribution de l'écureil roux (gauche) et de l'écureuil gris (droit).}
\end{figure}

\end{frame}
}

%% Interaction biotiques: effet d'une espèce sur le fitness (taux de croissance) d'une autre espèce
%% Historiquement: Conditions climatiques importantes à large échelle, et les interactions à l'échelle locale
%% Little interest in the study of biotic interactions at large geographical scales: important need for methods to account for interactions in future predictions

%%%%%%%%%%%%%%%%%%%%%%%%%%%%%%%%%%%%%%%%%%%%%%%%%%%%%%%%%%%%%%%%%%%%%%%%%%%%%%%%%
% % %

{%
\setbeamertemplate{frame footer}{Pulliam 2000, Godsoe \textit{et al}. 2017}
\begin{frame}{Théorie\hskip 1em \mdseries{\textcolor{lightgray}{La niche écologique}}}
  La \textbf{niche fondamentale} fait référence aux \emph{conditions climatiques}: là où l'espèce \emph{peut} être présente
  \begin{equation*}
    \underbrace{r(E)}_{\text{croissance}} > \underbrace{b(E)}_{\text{naissances}} - \underbrace{d(E)}_{\text{mortalité}}
  \end{equation*}

  \begin{columns}[]
    \begin{column}{0.5\textwidth}
      \begin{figure}
        \includegraphics[width=0.8\textwidth]{figs/puliam1.png}
      \end{figure}
    \end{column}

    \begin{column}{0.5\textwidth}
      \includegraphics[width=0.95\textwidth]{figs/godsoe1.png}
    \end{column}
  \end{columns}

\end{frame}
}

%% Godsoe: approche les limites de distribution avec la théorie de la niche (r>b-d)

%%%%%%%%%%%%%%%%%%%%%%%%%%%%%%%%%%%%%%%%%%%%%%%%%%%%%%%%%%%%%%%%%%%%%%%%%%%%%%%%%
% % %

{%
\setbeamertemplate{frame footer}{Pulliam 2000, Godsoe \textit{et al}. 2017}
\begin{frame}{Théorie\hskip 1em \mdseries{\textcolor{lightgray}{La niche écologique}}}

  %b,d --> c,e,h  (disponibilité d'habitat: montrer schéma que j'ai fait)
  La \textbf{niche réalisée} fait référence aux \emph{conditions climatiques et autres facteurs}: là où l'espèce \emph{est} présente
  \begin{equation*}
    \underbrace{r(E)}_{\text{croissance}} > \underbrace{b(E)}_{\text{naissances}} - \underbrace{d(E)}_{\text{mortalité}} + \underbrace{I(E)}_{\text{interactions}}
  \end{equation*}

  \begin{columns}[]
    \begin{column}{0.5\textwidth}
      \begin{figure}
        \includegraphics[width=0.7\textwidth]{figs/pulliam2.png}
      \end{figure}
    \end{column}

    \begin{column}{0.5\textwidth}
      \includegraphics[width=.9\textwidth]{figs/godsoe2.png}
    \end{column}
  \end{columns}

\end{frame}
}

%% Besoin d'une perspective régionale (\lambda>c-e --> suget de ma maitrise (montrer les schémas que j'ai fait)

%%%%%%%%%%%%%%%%%%%%%%%%%%%%%%%%%%%%%%%%%%%%%%%%%%%%%%%%%%%%%%%%%%%%%%%%%%%%%%%%%
% % %

{%
\setbeamertemplate{frame footer}{Schooley and Consentino 2018}
\begin{frame}{Théorie\hskip 1em \mdseries{\textcolor{lightgray}{Métapopulations}}}

  \begin{columns}
    \only<1-2>{%
      \begin{column}{0.5\textwidth}
        \begin{figure}
          \includegraphics[width=.60\paperheight]{figs/metapop.png}
          \caption*{Une métapopulation}
        \end{figure}
      \end{column}
    }

    \only<1>{%
      \begin{column}{0.5\textwidth}
      \end{column}
    }

    \only<2>{%
      \begin{column}{0.5\textwidth}
        \begin{equation*}
            \frac{dP}{dt} = c(h-P) - eP
        \end{equation*}
      \end{column}
    }
  \end{columns}

\end{frame}
}

%% Besoin d'une perspective régionale (\lambda>c-e --> suget de ma maitrise (montrer les schémas que j'ai fait)
%% h: disponibilité d'habitat --> patch habitable maintenant, mais peut être pas dans le futur à cause des c.c.
%% h>e/c --> condition de la distribution d'une espèce; h=e/c ou c*h=e est la limite de distribution

%%%%%%%%%%%%%%%%%%%%%%%%%%%%%%%%%%%%%%%%%%%%%%%%%%%%%%%%%%%%%%%%%%%%%%%%%%%%%%%%%
% % %

\begin{frame}{Théorie\hskip 1em \mdseries{\textcolor{lightgray}{Métapopulations}}}

  \begin{equation*}
    \textcolor{lightgray}{\underbrace{b(E)}_{\text{naissances}} = \underbrace{d(E)}_{\text{mortalité}}}
  \end{equation*}

  \begin{equation*}
    \underbrace{c(E)}_{\text{colonisations}} * \underbrace{h(E)}_{\text{disponibilité d'habitat}} = \underbrace{e(E)}_{\text{extinctions}}
  \end{equation*}

\end{frame}

%% Une population est une patch individuelle, une métapopulation est un amalgame de plusieurs patchs

%%%%%%%%%%%%%%%%%%%%%%%%%%%%%%%%%%%%%%%%%%%%%%%%%%%%%%%%%%%%%%%%%%%%%%%%%%%%%%%%%
% % %

\begin{frame}{Théorie\hskip 1em \mdseries{\textcolor{lightgray}{Métapopulations}}}

  La \textbf{capacité de support} est une mesure de la structure de l'habitat, c'est un proxy de la \emph{viabilité}

  \begin{figure}
    \includegraphics[width=.60\paperheight]{figs/metapop2.png}
    \caption*{Une métapopulation}
  \end{figure}

\end{frame}

%% La capacité de support est en quelque sorte une mesure du risque d'extinction de l'espèce; c'est une mesure de la structure de l'habitat (# patch, aire des patch; distances) tel qu'expérimenté par l'espèce sur sa persistence
%% Intéressent pour mesurer l'impact des changements climatiques sur une espèce à l'échelle régionale; différence dans la viabilité
%% Intéressent pour mesurer l'impact des interactions sur la réponse espèce à l'échelle régionale; différence dans le risque d'extinction avant-après

%%%%%%%%%%%%%%%%%%%%%%%%%%%%%%%%%%%%%%%%%%%%%%%%%%%%%%%%%%%%%%%%%%%%%%%%%%%%%%%%%
% % %

\begin{frame}{Étude de cas\hskip 1em \mdseries{\textcolor{lightgray}{Les Appalaches}}}

  \begin{figure}
  	\includegraphics[height=.65\paperheight]{figs/appalaches.png} % eg. grive et sa foret
    %\caption*{ \textbf{}}
  \end{figure}

\end{frame}

%% La distribution à l'échelle régionale
%% Le passereau est dépendant des conditions climatiques et de son habitat
%% Son habitat est en retard sur les changements climatiques = mismatch

%%%%%%%%%%%%%%%%%%%%%%%%%%%%%%%%%%%%%%%%%%%%%%%%%%%%%%%%%%%%%%%%%%%%%%%%%%%%%%%%%
% % %

\begin{frame}{Objectifs\hskip 1em \mdseries{\textcolor{lightgray}{Subtitle}}}

  \textbf{Objectif général:} Évaluer les impacts des changements climatiques sur la distribution en Estrie de la grive de Bicknell.

  \textbf{\\Objectifs secondaires:}
  \begin{enumerate}
    \item Développer un nouvel outil théorique pour améliorer la prédiction des impacts des changements climatiques sur la distribution de la grive de Bicknell;
    \item Évaluer l'impact des interactions biotiques sur la réaction des aires de distribution aux changements climatiques.
  \end{enumerate}

\end{frame}

%%%%%%%%%%%%%%%%%%%%%%%%%%%%%%%%%%%%%%%%%%%%%%%%%%%%%%%%%%%%%%%%%%%%%%%%%%%%%%%%%
% % %

\begin{frame}{Approche\hskip 1em \mdseries{\textcolor{lightgray}{Développer un outil théorique}}}

  1. Schématiser le problème

$$h(T) > \frac{e(T)}{c(T)}$$

  \begin{figure}
    \only<1>{\includegraphics[height=.50\paperheight]{figs/modele1.png}}
    \only<2>{\includegraphics[height=.50\paperheight]{figs/modele2.png}}
    \only<3>{\includegraphics[height=.50\paperheight]{figs/modele3.png}}
  \end{figure}

\end{frame}

%% L'espèce à l'échelle régionale sera présente là où le ratio est suppérieur à la disponibilité de l'habitat (sa forêt)
%% Amène à une contraction de l'aire de distribution pour une espèce de sommet
%% Amène à une extension de l'aire de distribution pour un espèce associée à la base

%%%%%%%%%%%%%%%%%%%%%%%%%%%%%%%%%%%%%%%%%%%%%%%%%%%%%%%%%%%%%%%%%%%%%%%%%%%%%%%%%
% % %

\begin{frame}{Approche\hskip 1em \mdseries{\textcolor{lightgray}{Développer un outil théorique}}}

  1. Schématiser le problème.

  2. Traduire le problème en un \textbf{modèle de métapopulation} qui tiendra compte de \emph{la disponibilité d'habitat}, de \emph{la structure du paysage} et \emph{du climat}.

  %\begin{equation*}
  %  \frac{dP_i}{dt} = c(T_i) \sum{exp(-\alpha d_ij)} f(T_j,A_j)(1-P_i) - e(T_i,A_i)P_i
  %\end{equation*}

  \begin{figure}
    \includegraphics[height=.55\paperheight]{figs/appalaches.png}
  \end{figure}

\end{frame}

%% Modèle réalistique (spatiallement explicite)
%% Une équation pour la probabilité que chaque patch soit occuppée
%% Prend en compte la distribution spatiale des patchs (montagnes) (aire, distance), le climat de chaque patch et la dispersion de l'espèce
%% Le modèle Prédira la distribution de l'espèce après un changement du climat

%%%%%%%%%%%%%%%%%%%%%%%%%%%%%%%%%%%%%%%%%%%%%%%%%%%%%%%%%%%%%%%%%%%%%%%%%%%%%%%%%
% % %

\begin{frame}[t]{Approche\hskip 1em \mdseries{\textcolor{lightgray}{L'impact des interactions}}}

  1. Schématiser le problème.

  2. Traduire le problème en un \textbf{modèle de métapopulation} qui tiendra compte de \emph{la disponibilité d'habitat}, de \emph{la structure du paysage} et \emph{du climat}.

  3. Mesurer la \textbf{capacité de support} de la métapopulation.

  % Figure du range (métapop) de l'espèce avant et après changements climatiques

\end{frame}

%% Objectif: mesurer l'impact sur l'ampleur de la réponse à l'échelle régionale (à l'échelle de la métapop)
%% En regardant le degré de réponse de l'habitat, l'effet des interactions sur la réponse pourra être observée sur le risque d'extinction de l'espèce apres changements climatiques

%%%%%%%%%%%%%%%%%%%%%%%%%%%%%%%%%%%%%%%%%%%%%%%%%%%%%%%%%%%%%%%%%%%%%%%%%%%%%%%%%
% % %

\begin{frame}[t]{Approche\hskip 1em \mdseries{\textcolor{lightgray}{L'impact des interactions}}}

  1. Schématiser le problème.

  2. Traduire le problème en un \textbf{modèle de métapopulation} qui tiendra compte de \emph{la disponibilité d'habitat}, de \emph{la structure du paysage} et \emph{du climat}.

  3. Mesurer la \textbf{capacité de support} de la métapopulation.

  4. Mesurer la \textbf{vitesse de réaction} de la métapopulation

\end{frame}

%% Objectif: mesurer l'impact sur la dynamique de la réponse; de la sensibilité de l'espèce auc c.c.

%%%%%%%%%%%%%%%%%%%%%%%%%%%%%%%%%%%%%%%%%%%%%%%%%%%%%%%%%%%%%%%%%%%%%%%%%%%%%%%%%
% % %

\begin{frame}{Contributions\hskip 1em \mdseries{\textcolor{lightgray}{}}}

  Mon travail contribuera à:

  \begin{itemize}
    \item Une meilleure compréhension de \textbf{l'effet des interactions} sur la réponse des espèces face au changements climatiques;
    \item Produira des \textbf{hypothèses} qui pourront être vérifiées sur le terrain;
    \item Aidera Corridor Appalachien à \textbf{documenter le futur} des sommets de conifères et de certaines espèces.
  \end{itemize}

\end{frame}

%% Mon travail amènera une meilleure compréhension de l'effet des interactions sur la réponse des espèces face aux changements climatiques qui était ingnorée des modèles de distribution d'espèce classiques alors qu'il y a d'importantes preuves de leur effet.
%% Prédiction des effets des changements climatiques sur la distribution plus précises

%%%%%%%%%%%%%%%%%%%%%%%%%%%%%%%%%%%%%%%%%%%%%%%%%%%%%%%%%%%%%%%%%%%%%%%%%%%%%%%%%
% % %

\begin{frame}{Remerciements\hskip 1em \mdseries{\textcolor{lightgray}{}}}

  \begin{columns}
    \begin{column}{0.3\textwidth}
      \textbf{Directeur de recherche}
      Dominique Gravel
      \par
      \vspace{1em}
      \textbf{Commité de conseillers}
      Mark Vellend

      Marc Bélisle

      Anna Hargreaves
      \par
      \vspace{1em}
      Membres du laboratoire Gravel
      \par
      \vspace{1em}
      Guillaume blanchet
    \end{column}

    \begin{column}{0.7\textwidth}
      \par
      \vspace{1em}
      \includegraphics[width=\textwidth]{figs/corridor.jpg}
    \end{column}
  \end{columns}

\end{frame}

%%%%%%%%%%%%%%%%%%%%%%%%%%%%%%%%%%%%%%%%%%%%%%%%%%%%%%%%%%%%%%%%%%%%%%%%%%%%%%%%%
% % %

\begin{frame}[standout]

  Questions?

\end{frame}

%%%%%%%%%%%%%%%%%%%%%%%%%%%%%%%%%%%%%%%%%%%%%%%%%%%%%%%%%%%%%%%%%%%%%%%%%%%%%%%%%
%%%%%%%%%%%%%%%%%%%%%%%%%%%%%%%%%%%%%%%%%%%%%%%%%%%%%%%%%%%%%%%%%%%%%%%%%%%%%%%%%
%%%%%%%%%%%%%%%%%%%%%%%%%%%%%%%%%%%%%%%%%%%%%%%%%%%%%%%%%%%%%%%%%%%%%%%%%%%%%%%%%
% % %
\appendix

\begin{frame}{Appendice\hskip 1em \mdseries{\textcolor{lightgray}{Modèle mathématique}}}{Backup slides}

  Modèle spatiallement explicite de métapopulation

  \begin{equation*}
    \frac{dP_i}{dt} = c(T_i) \sum{exp(-\alpha d_ij)} f(T_j,A_j)(1-P_i) - e(T_i,A_i)P_i
  \end{equation*}

  \begin{figure}
    \includegraphics[height=.50\paperheight]{figs/grive.png}
  \end{figure}

\end{frame}

%%%%%%%%%%%%%%%%%%%%%%%%%%%%%%%%%%%%%%%%%%%%%%%%%%%%%%%%%%%%%%%%%%%%%%%%%%%%%%%%%
% % %

{%
\setbeamertemplate{frame footer}{Svenning \textit{et al}. 2014}
\begin{frame}{Appendice\hskip 1em \mdseries{\textcolor{lightgray}{Range shifts}}}

  \begin{figure}
  	\includegraphics[height=.7\paperheight]{figs/svenning.png}
    %\caption*{ \textbf{}}
  \end{figure}

\end{frame}
}

%%%%%%%%%%%%%%%%%%%%%%%%%%%%%%%%%%%%%%%%%%%%%%%%%%%%%%%%%%%%%%%%%%%%%%%%%%%%%%%%%
% % %

{%
\setbeamertemplate{frame footer}{}
\begin{frame}{Appendice\hskip 1em \mdseries{\textcolor{lightgray}{Range shifts}}}

  \begin{figure}
  	\includegraphics[height=.7\paperheight]{figs/fut_range.pdf}
  \end{figure}

\end{frame}
}

%%%%%%%%%%%%%%%%%%%%%%%%%%%%%%%%%%%%%%%%%%%%%%%%%%%%%%%%%%%%%%%%%%%%%%%%%%%%%%%%%
% % %

{%
\setbeamertemplate{frame footer}{}
\begin{frame}{Appendice\hskip 1em \mdseries{\textcolor{lightgray}{Résumé}}}

  \textbf{Q:} Quel est l'impact des interactions sur la réponse des espèces face aux changements climatiques

  \begin{enumerate}
    \item Les interactions peuvent affecter la distribution à l'échelle régionale
    \item La forêt accusera un retard sur le déplacement des conditions climatiques
    \item Les passereaux ont le potentiel de suivre le déplacement des conditions climatiques, mais ont besoin de leur habitat
  \end{enumerate}

\end{frame}
}

%%%%%%%%%%%%%%%%%%%%%%%%%%%%%%%%%%%%%%%%%%%%%%%%%%%%%%%%%%%%%%%%%%%%%%%%%%%%%%%%%

\end{document}
